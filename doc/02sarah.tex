%%%%%%%%%%%%%%%%%%%%%%%%%%%%%%%%%%%%%%%%%%%%%%%%%%%%%%%%%%%%%%%%%%%%%%%%%%%%%%%%%%
%%													
							%%
%% File name: 		20hannes.tex									
			%%
%% Project name:	Hochleistungsantenne								
		%%
%% Type of work:	T3X00 project work								
			%%
%% Author:			Sarah Brückner, Maximilian Stiefel, Hannes Bohnengel		%%
%% Date:			30th May 2016								
				%%
%% University:		DHBW Ravensburg Campus Friedrichshafen						
%%
%% Comments:		Created in gedit with tab width = 4						
	%%
%%													
							%%
%%%%%%%%%%%%%%%%%%%%%%%%%%%%%%%%%%%%%%%%%%%%%%%%%%%%%%%%%%%%%%%%%%%%%%%%%%%%%%%%%%


\chapter{Aufbau der Bodenstation}
\label{chap:bodenstation}

Die Bodenstation der DHBW Ravensburg am Standort Friedrichshafen besteht grundsätzlich aus folgenden 
Komponenten:

\begin{itemize}
	\parskip0pt
	\item Antennen (drei Bänder)
	\begin{itemize}
		\item Endstufen / Verstärker
		\item Mast (Steuergerät für Höheneinstellung)
	\end{itemize}
	\item Rotoren
	\begin{itemize}
		\item Banana Pi + interne Software
		\item ARSVCOM Software
		\item Azimut-Rotor + Steuergerät
		\item Elevations-Rotor + Steuergerät
	\end{itemize}
	\item Funkgerät Icom IC-9100
	\begin{itemize}
		\item Netcom (2x Seriell zu Ethernet)
		\item Netcom Manager Software
		\item Hardware für Sprechfunk ?!
	\end{itemize}
\end{itemize}

\begin{center}
	\Large{\textbf{--- !!! ---}\\Liste an aktiven Amateurfunksatelliten in Abschnitt Amateurfunk 
einfügen\\\textbf{--- !!! ---}}
\end{center} 
