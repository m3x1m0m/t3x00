%%%%%%%%%%%%%%%%%%%%%%%%%%%%%%%%%%%%%%%%%%%%%%%%%%%%%%%%%%%%%%%%%%%%%%%%%%%%%%%%%%
%%																				%%
%% File name: 		abstract.tex												%%
%% Project name:	Hochleistungsantenne										%%
%% Type of work:	T3X00 project work											%%
%% Author:			Sarah Brückner, Maximilian Stiefel, Hannes Bohnengel		%%
%% Date:			11th July 2016												%%
%% University:		DHBW Ravensburg Campus Friedrichshafen						%%
%% Comments:		Created in gedit with tab width = 4							%%
%%																				%%
%%%%%%%%%%%%%%%%%%%%%%%%%%%%%%%%%%%%%%%%%%%%%%%%%%%%%%%%%%%%%%%%%%%%%%%%%%%%%%%%%%

\chapter*{Kurzfassung}
Diese Studienarbeit beschreibt die Inbetriebnahme einer freien Software zur Satellitenbahnvorhersage und Ansteuerung einer Hochleistungsantenne. 
Dies schließt die Nachführung der Antenne und damit die Steuerung von Rotoren mit ein. Die freie Software soll die bestehende Software 
SatPC32 ersetzen. SatPC32 ist eine proprietäre Software und dient der Bodenstation als Referenz. Da SatPC32 nicht mehr weiterentwickelt wird 
und es darüber hinaus keinen Support für diese Software gibt, wurde der alternative Weg, einer freien Software für die Bodenstation in 
Friedrichshafen, gewählt. Diese freie Software ist GPredict. GPredict ist quelloffen und beherrscht es mit Hilfe der \ac{TLE} die Satellitenposition 
vorherzusagen und diese auf einer grafischen Oberfläche anzuzeigen. Des Weiteren wird von GPerdict bzw. der darunter liegenden HamLib eine Vielzahl 
verschiedener Hardware-Einheiten zur Ausrichtung der Antenne und zur Kommunikation unterstützt. Eine Kommunikation mit Transpondern im Orbit 
aufzubauen ist somit durch GPredict realisierbar.
 
\vspace{3em}
\begin{Huge}
	\textbf{Abstract}
\end{Huge}
\vspace{1.5em}

Translation of ,,Kurzfassung'' comes here...

\clearpage
