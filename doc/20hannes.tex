%%%%%%%%%%%%%%%%%%%%%%%%%%%%%%%%%%%%%%%%%%%%%%%%%%%%%%%%%%%%%%%%%%%%%%%%%%%%%%%%%%
%%																				%%
%% File name: 		20hannes.tex												%%
%% Project name:	Hochleistungsantenne										%%
%% Type of work:	T3X00 project work											%%
%% Author:			Sarah Brückner, Maximilian Stiefel, Hannes Bohnengel		%%
%% Date:			24th May 2016												%%
%% University:		DHBW Ravensburg Campus Friedrichshafen						%%
%% Comments:		Created in gedit with tab width = 4							%%
%%																				%%
%%%%%%%%%%%%%%%%%%%%%%%%%%%%%%%%%%%%%%%%%%%%%%%%%%%%%%%%%%%%%%%%%%%%%%%%%%%%%%%%%%

\chapter{Aufbau der Bodenstation}

Die Bodenstation der DHBW Ravensburg am Standort Friedrichshafen besteht grundsätzlich aus folgenden Komponenten:

\begin{itemize}
	\parskip0pt
	\item Antennen (drei Bänder)
	\begin{itemize}
		\item Endstufen / Verstärker
		\item Mast (Steuergerät für Höheneinstellung)
	\end{itemize}
	\item Rotoren
	\begin{itemize}
		\item Banana Pi + interne Software
		\item ARSVCOM Software
		\item Azimut-Rotor + Steuergerät
		\item Elevations-Rotor + Steuergerät
	\end{itemize}
	\item Funkgerät Icom IC-9100
	\begin{itemize}
		\item Netcom (2x Seriell zu Ethernet)
		\item Netcom Manager Software
		\item Hardware für Sprechfunk ?!
	\end{itemize}
\end{itemize}


\chapter{GPredict}

\section{Übersicht}

GPredict ist eine freie Software zur Satellitenverfolgung und Orbitvorhersage und steht als Quellcode oder bereits fertig kompiliertes Programm für Windows, Mac OS und Linux zur Verfügung. Die Software ist in C geschrieben und unter der GNU \ac{GPL} lizenziert, somit kann sie frei verändert und an die entsprechenden Nutzervoraussetzungen angepasst werden.\newpar
In Abbildung \ref{fig:gpredict-principle} ist das Prinzip eines Satellitenverfolgungsprogramms zu sehen (die blauen Blöcke stellen hierbei die Funktionalität des Programms dar). Zunächst wird an Hand der Keplerschen Bahnelemente und dem aktuellen Zeitpunkt die absolute Position des Satelliten berechnet. Daraufhin wird der Vektor, der von der Bodenstation zum Satelliten zeigt, bestimmt. Nun können Azimut und Elevation dieses Vektors für die Ansteuerung der Antenne verwendet werden.

\begin{figure}[h]
	\centering
	\includegraphics[width=1\textwidth]{gpredict-principle}
	\caption{Prinzip eines Satellitenverfolgungsprogramms, Quelle: \cite{gpredictmanual}}
	\label{fig:gpredict-principle} 
\end{figure}

Zur Berechnung der Satellitenposition wird auf den NORAD SGP4/SDP4 Algorithmus zurückgegriffen (siehe Abschnitt XXX). Um hierfür zu jedem Zeitpunkt die aktuellen Kepler-Elemente des zu verfolgenden Satelliten zu kennen, gibt es unter GPredict die Möglichkeit einer automatischen Aktualisierung über HTTP, FTP oder aus dem lokalen Verzeichnis.

\clearpage

Bei GPredict ist im Gegensatz zu anderen Satellitenverfolgungsprogrammen wie SatPC32 kein Limit an zu verfolgenden Satelliten und Bodenstationen gegeben. Durch die Verwendung von Modulen kann außerdem unkompliziert zwischen verschiedenen Konfigurationen gewechselt werden. Die Orbitvorhersage eines Satelliten lässt sich sowohl grafisch als auch tabellarisch darstellen, wobei durch die Einstellungen verschiedenster Parameter eine sehr individuelle Anzeige erreicht werden kann \cite{gpredictsource}.

\section{Grafische Oberfläche}

In Abbildung \ref{fig:gpredictstartup} ist die grafische Oberfläche von GPredict zu sehen. In der Standardkonfiguration ist dort zunächst die Kartenansicht bzw. \myemph{Map View} (oben), die Polaransicht bwz. \myemph{Polar View} (links unten) und die Einzelsatellitenansicht bzw. \myemph{Single-Satellite View} (rechts unten) zu sehen.

\begin{figure}[h]
	\centering
	\includegraphics[width=0.75\textwidth]{gpredict-startup}
	\caption{Standardoberfläche von GPredict}
	\label{fig:gpredictstartup} 
\end{figure}

\clearpage

\subsection{Grundansichten}

Zu den oben genannten Ansichten kommen noch zwei Weitere hinzu, die Listenansicht bzw. \myemph{List View} und eine Ansicht für bevorstehende Vorbeiflüge, die sogenannte \myemph{Upcoming Passes View}. Im Folgenden werden die verschiedenen Ansichten genauer beschrieben:\newpar
\textbf{Map View}\\
Diese Ansicht besteht, wie in Abbildung \ref{fig:gpredictstartup} zu sehen, aus einer Weltkarte auf der die aktuellen Standorte der für das aktuelle Modul ausgewählten Satelliten zu sehen ist. Das heißt der Punkt auf dem der entsprechende Satellit senkrecht bezogen auf den Erdmittelpunkt steht. Außerdem ist um diesen Punkt die Fläche umrahmt, von der der Satellit von der Erde aus sichtbar ist. Mit einem Rechtsklick auf einen Satellitennamen kann außerdem die Option \myemph{Ground Track} aktiviert werden, mit welcher die Spur des Satelliten für mehrere Orbits angezeigt wird.\newpar
\textbf{Polar View}\\
Die \myemph{Polar View} (siehe Abbildung \ref{fig:gpredictstartup}) stellt eine Draufsicht auf die Bodenstation dar, bei der die Polarachse den Azimutwinkel darstellt und die Radialachse den Elevationswinkel. Mit einem Rechtsklick auf einen Satelliten lässt sich mit der Option \myemph{Show sky track} aktivieren, das die Spur des entsprechenden Satelliten anzeigt wird. Zusätzlich wird das aktuelle Modul links oben angezeigt, der nächste sichtbare Satellit (rechts oben) und die genauen Werte für Azimut und Elevation (links unten) sobald sich der Mauszeiger auf der \myemph{Polar View} befindet.\newpar
\textbf{Single-Satellite View}\\
In dieser Ansicht (siehe Abbildung \ref{fig:gpredictstartup}) werden detaillierte Informationen zu einem ausgewählten Satelliten angezeigt, z.B. Azimut, Elevation, Entfernung der direkten Sichtverbindung (\myemph{Slant Range}), Höhe, Geschwindigkeit, Dopplerverschiebung oder Signaldämpfung. Mit einem Klick auf das \myvsymbol-Symbol links neben dem Satellitennamen kann zwischen den für dieses Modul ausgewählten Satelliten gewechselt werden.

\clearpage

\textbf{List View}\\
Die Listenansicht zeigt eine tabellarische Auflistung aller für das aktuelle Modul ausgewählten Satelliten mit verschiedenen Details, mit je einem Satelliten pro Zeile. In Abbildung \ref{fig:listview} ist die Listenansicht mit allen verfügbaren Details zu sehen. Mit einem Klick auf eine entsprechende Kategorie lässt sich das Sortierkriterium ändern. Falls hier ein variables Kriterium wie die Geschwindigkeit eingestellt wird, ändert sich die Sortierreihenfolge mit der eingestellten Auffrischrate (\myemph{Refresh Rate}). Die Bezeichnung des jeweiligen Details ist in dieser Ansicht abgekürzt, z.B. \myemph{Az} für \myemph{Azimut}. Unter den Moduleinstellungen beim Reiter \myemph{List View} kann ausgewählt werden, welches Detail angezeigt wird. Dort ist außerdem erkenntlich für was die entsprechenden Abkürzungen stehen.

\begin{figure}[h]
	\centering
	\includegraphics[width=1\textwidth]{listview}
	\caption{Listenansicht bzw. \myemph{List View} von GPredict}
	\label{fig:listview} 
\end{figure}

\textbf{Upcoming Passes View}\\
Die \myemph{Upcoming Passes View} (siehe Abbildung \ref{fig:upcomingpassesview}) zeigt alle Satelliten des aktuellen Moduls, deren Azimut und Elevation, sowie die Zeit bis zum nächsten Verschwinden des Satelliten, dem sogenannten \myemph{\ac{LOS}} bzw. dem nächsten Auftauchen, auch \myemph{\ac{AOS}} genannt. Wie bei der \myemph{List View} ist es auch hier möglich nach den verschiedenen Spalten zu sortieren.

\begin{figure}[h]
	\centering
	\includegraphics[width=0.4\textwidth]{upcomingpassesview}
	\caption{Upcoming Passes View}
	\label{fig:upcomingpassesview} 
\end{figure}

\clearpage

\subsection{Weitere Ansichten}

Bei allen Ansichten kann durch einen Klick auf den Satellitennamen ein kleines Pop-Up Menü geöffnet werden, welches den entsprechenden Satellitennamen, die Option \myemph{Show next pass} und die Option \myemph{Future passes} anzeigt. Bei einem Klick auf den Satellitennamen öffnet sich ein Fenster mit dem Titel \myemph{Satellite Info}, wie in Abbildung \ref{fig:satinfo} zu sehen. Dort sind unter dem Reiter \myemph{Orbit Info} verschiedene Informationen zum Satellitenorbit und unter dem Reiter \myemph{Transponders} die verfügbaren Transponder zu sehen.

% (bei der \myemph{Single-Satellite View} ein Klick auf das Dreieck neben dem Namen)
% (oder einen Doppelklick in der entsprechenden Ansicht auf den Satellitennamen)

\begin{figure}[h]
	\centering
	\includegraphics[width=0.65\textwidth]{satinfo}
	\caption{Satellite Info}
	\label{fig:satinfo} 
\end{figure}

Mit einem Klick auf die Option \myemph{Show next pass} gelangt man zu einer Übersicht über den nächsten Vorbeiflug des entsprechenden Satelliten. Die Details sind tabellarisch, als Polaransicht und als Verlauf des Azimut- und Elevationswinkels über der Zeit zu sehen (siehe Abbildung \ref{fig:passdetails}).

\begin{figure}[h]
	\centering
	\includegraphics[width=1\textwidth]{passdetails}
	\caption{Pass Details}
	\label{fig:passdetails} 
\end{figure}

\clearpage

Die Option \myemph{Future passes} öffnet ein Fenster, in welchem die nächsten Vorbeiflüge des entsprechenden Satelliten tabellarisch dargestellt sind (siehe Abbildung \ref{fig:upcomingpasses}). Hierbei ist die Anzahl der darzustellenden Vorbeiflüge in den GPredict-Einstellungen unter \myemph{Predict} als \myemph{Number of passes to predict} einstellbar.

\begin{figure}[h]
	\centering
	\includegraphics[width=0.5\textwidth]{upcomingpasses}
	\caption{Upcoming Passes}
	\label{fig:upcomingpasses} 
\end{figure}

\vspace{-0.5em}

\subsection{Modul Pop-Up Menü}

Um das Modul Pop-Up Menü zu öffnen, klickt man ganz rechts oben im GPredict-Fenster auf das \myvsymbol-Symbol. Im daraufhin erscheinenden Pop-Up Menü ist es möglich die Positionierung eines Moduls innerhalb des GPredict-Fensters einzustellen, ein Modul zu kopieren, zu löschen, zu schließen oder genauer zu konfigurieren. Außerdem sind dort weitere Funktionen, welche im Folgenden genauer beschrieben werden, zugänglich.\newpar
Wie in Abbildung \ref{fig:theskyataglance} zu sehen, bietet die Funktion \myemph{Sky at a glance} eine Übersicht darüber, wann welche Satelliten innerhalb der nächsten acht Stunden sichtbar sind. Dieser Zeitraum lässt sich in den GPredict-Einstellungen bei \myemph{Predict} unter dem Reiter \myemph{Sky at a Glance} zwischen einer und 24 Stunden einstellen.

\begin{figure}[h]
	\centering
	\includegraphics[width=0.5\textwidth]{theskyataglance}
	\caption{The sky at a glance}
	\label{fig:theskyataglance} 
\end{figure}

\clearpage

Über die Funktion \myemph{Time Controller} (siehe Abbildung \ref{fig:timecontroller}) lässt sich die Zeit, auf die sich die Berechnungen von GPredict beziehen, ändern. Hierbei ist standardmäßig das aktuelle Datum und die aktuelle Uhrzeit eingestellt. Außerdem kann hier die Geschwindigkeit, mit der die eingestellte Zeit fortschreitet, auf maximal ein Hundertfaches erhöht werden. Die eingestellte Zeit wird im GPredict-Fenster ganz links oben im ausgewählten Format angezeigt. Mit dem Schieberegler kann die Zeit zwischen --2,5 und +2,5 Stunden eingestellt werden.

\begin{figure}[h]
	\centering
	\includegraphics[width=0.35\textwidth]{timecontroller}
	\caption{Time Controller}
	\label{fig:timecontroller} 
\end{figure}

Klickt man auf \myemph{Configure}, öffnet sich ein Fenster wie in Abbildung \ref{fig:editmodule} zu sehen. Hier lassen sich die zu verfolgenden Satelliten und die Bodenstation für das aktuelle Modul auswählen. Außerdem gelangt man mit einem Klick auf das Feld \myemph{Properties} in die Modul-Einstellungen. Diese gelten im Gegensatz zu den in den allgemeinen Einstellungen zu findenden Modul-Einstellungen nur für das aktuelle Modul.


\begin{figure}[h]
	\centering
	\includegraphics[width=0.5\textwidth]{editmodule}
	\caption{Moduleinstellungen}
	\label{fig:editmodule} 
\end{figure}

\clearpage

Hinter der Funktion \myemph{Antenna Control} (siehe Abbildung \ref{fig:rotatorcontrol}) verbirgt sich ein Bedienfeld zur Steuerung der Antennenrotoren. Bevor dieses geöffnet werden kann, muss zunächst in den GPredict-Einstellungen unter \myemph{Interfaces} mindestens eine Schnittstelle zur Rotorsteuerung konfiguriert werden (siehe Abschnitt XXX). 
\begin{figure}[h]
	\centering
	\includegraphics[width=0.6\textwidth]{rotatorcontrol-active}
	\caption{Rotorsteuerungs-Bedienfeld}
	\label{fig:rotatorcontrol} 
\end{figure}

Das Bedienfeld beinhaltet zum Einen eine Polaransicht, auf der die Spur und der aktuelle Ort des zu verfolgenden Satelliten (dargestellt durch ein Viereck) und die gegenwärtige Ausrichtung der Antenne (dargestellt durch ein Fadenkreuz) zu sehen ist. Zum Anderen sind folgende vier Bereiche verfügbar:

\begin{itemize}
	\parskip0pt
	\item \textbf{Azimuth:} In diesem Feld lässt sich die Ausrichtung der Antenne in Azimut-Richtung steuern, vorausgesetzt, dass die \myemph{Track}-Funktion nicht aktiviert ist. Am unteren Ende des Feldes wird unter \myemph{Read} der aktuelle Azimut-Winkel der Antenne angezeigt. Ist keine Verbindung zum Rotor aufgebaut, wird hier ,,-\,-\,-'' angezeigt. Liegt ein Verbindungsproblem vor, erscheint ,,ERROR''.
	\item \textbf{Elevation:} In diesem Feld lässt sich die Ausrichtung der Antenne in Elevations-Richtung steuern, vorausgesetzt, dass die \myemph{Track}-Funktion nicht aktiviert ist. Am unteren Ende des Feldes wird unter \myemph{Read} der aktuelle Elevations-Winkel der Antenne angezeigt. Ist keine Verbindung zum Rotor aufgebaut, wird hier ,,-\,-\,-'' angezeigt. Liegt ein Verbindungsproblem vor, erscheint ,,ERROR''.
	\item \textbf{Target:} Hier lässt sich der zu verfolgende Satellit auswählen. Es stehen hierbei nur die für das aktuelle Modul ausgewählten Satelliten zu Verfügung. Aktiviert man die Schaltfläche \myemph{Track}, wird der ausgewählte Satellit verfolgt. Unter dem Satellitennamen werden die jeweiligen Winkel in Echtzeit dargestellt und hinter $\Delta$T wird die Zeit bis zum nächsten \ac{AOS} bzw. \ac{LOS} angezeigt.
	\clearpage
	\item \textbf{Settings:} Hier lässt sich die in den GPredict-Einstellungen festgelegte Schnittstelle zur Kommunikation mit den Rotoren auswählen. Mit einem Klick auf die Schaltfläche \myemph{Engage} wird die Verbindung zu dieser Schnittstelle aufgebaut bzw. unterbrochen. Unter \myemph{Cycle} kann dabei der Zyklus eingestellt werden, in welchem Kommandos an die Rotor-Schnittstelle gesendet und Winkelwerte von dieser abgefragt werden. Ein sinnvoller Wert liegt hierbei zwischen zwei und fünf Sekunden. Bei \myemph{Tolerance} wird die tolerierte Differenz zwischen abgefragtem und eingestelltem Winkel eingetragen. Sobald diese überschritten wird, wird ein  Kommando an die Rotor-Schnittstelle geschickt. Hierbei sollte sowohl die Winkelauflösung der Rotorsteuerung, als auch die Keulenbreite der Antenne berücksichtigt werden. Nach fünf aufeinanderfolgenden Fehlern bei der Kommunikation mit den Rotoren, wird die Verbindung automatisch unterbrochen.
\end{itemize}

In Tabelle \ref{tab:rotatorcontrolmodes} sind alle möglichen Kombinationen der Schaltflächen \myemph{Track} und \myemph{Engage} und deren Auswirkung beschrieben.

\begin{table}[h]
	\begin{tabularx}{\textwidth}{|l|l|X|}
		\hline
		\textbf{Track} 	    & \textbf{Engage}	&\textbf{Beschreibung}\\
		\hline
		Inaktiv          	& Inaktiv 			& Es werden weder Kommandos an die Rotoren gesendet, noch wird die aktuelle Ausrichtung der Antenne ausgelesen. Die aktuellen Winkel des zu verfolgenden Satellits werden nicht in die Winkelsteuerungs-Eingabefelder übertragen.\\
		Aktiv              	& Inaktiv   		& Die aktuellen Winkel des zu verfolgenden Satellits werden in die Winkelsteuerungs-Eingabefelder übertragen, es werden aber keine Kommandos an die Rotoren geschickt und die aktuelle Ausrichtung der Antenne wird nicht ausgelesen.\\
		Aktiv              	& Aktiv	            & Die aktuellen Winkel des zu verfolgenden Satellits werden in die Winkelsteuerungs-Eingabefelder übertragen und diese werden an die Rotoren geschickt. Die aktuelle Ausrichtung der Antenne wird ausgelesen.\\
		Inaktiv            	& Aktiv   			& Die Winkel, die in den Winkelsteuerungs-Eingabefelder eingestellt sind, werden an die Rotoren gesendet und die aktuelle Ausrichtung der Antenne wird ausgelesen.\\
		\hline		
	\end{tabularx}
	\caption{Betriebsmodi des \myemph{Antenna Control}-Bedienfelds, Quelle: \cite{gpredictmanual} \vspace{-2em}}
	\label{tab:rotatorcontrolmodes}
\end{table}

\clearpage

Um dem Funkgerät die entsprechenden Up- und Downlink-Frequenzen inklusive Dopplerschiftkorrektur zu übermitteln, wird das \myemph{Radio Control}-Bedienfeld (siehe Abbildung \ref{fig:radiocontrol}) verwendet. Dieses kann nur geöffnet werden, wenn mindestens eine Funkgerät-Schnittstelle in den GPredict-Einstellungen unter \myemph{Interfaces} konfiguriert ist (siehe Abschnitt XXX).

\begin{figure}[h]
	\centering
	\includegraphics[width=0.6\textwidth]{radiocontrol-active}
	\caption{Funkgerät-Steuerung}
	\label{fig:radiocontrol} 
\end{figure}

\begin{table}[h]
	\begin{tabularx}{\textwidth}{|l|l|X|}
		\hline
		\textbf{Track} 	    & \textbf{Engage}	&\textbf{Beschreibung}\\
		\hline
		Inaktiv          	& Inaktiv 			& Blabla\\
		Aktiv              	& Inaktiv   		& Blabla\\
		Aktiv              	& Aktiv	            & Blabla\\
		Inaktiv            	& Aktiv   			& Blabla\\
		\hline		
	\end{tabularx}
	\caption{Betriebsmodi des \myemph{Radio Control}-Bedienfelds, Quelle: \cite{gpredictmanual}}
	\label{tab:radiocontrolmodes}
\end{table}

\clearpage

\subsection{GPredict Einstellungen}

\begin{itemize}
	\parskip0pt
	\item \textbf{General}
	\item \textbf{Modules}
	\item \textbf{Interfaces}
	\item \textbf{Predict}
\end{itemize}

\clearpage

\section{HamLib-Programmierschnittstelle}

\subsection{Übersicht}

Da es keinen einheitlichen Kommunikationsstandard für die zahlreichen Funkgeräte und Rotoren unterschiedlicher Hersteller gibt, ist für die Verwendung von GPredict eine applikationsspezifische Programmierschnittstelle oder auch \ac{API} erforderlich. Mit den \myemph{Ham Radio Control Libraries} (englisch für Amateurfunk-Kontrollbibliotheken), kurz HamLib, steht dem Benutzer eine solche \ac{API} zur Verfügung. HamLib ist unter der \ac{GPL} lizenziert und steht als freie Software jedem zur Verfügung. Wie in Abbildung \ref{fig:hamlib} zu sehen, ermöglicht HamLib einer Software wie GPredict die Kommunikation mit verschiedenen Funkgeräten und Rotoren, in dem es für jedes dieser Geräte einen eigenen Treiber bzw. ein eigenes \ac{BE} zur Verfügung stellt.

\begin{figure}[h]
	\centering
	\includegraphics[width=0.5\textwidth]{hamlib}
	\caption{HamLib Design, Quelle: \cite{hamlib}}
	\label{fig:hamlib} 
\end{figure}

Dabei verwendet man entweder den Quellcode um eine benutzerspezifische Anwendung zu erstellen oder man greift auf die bereits fertig kompilierten Programme zurück, welche im Folgenden aufgelistet sind:

\begin{itemize}
	\parskip0pt
	\item \textbf{rigctl:} Ein Kommandozeilenprogramm, mit welchem man Befehle über die Kommandozeile an das Funkgerät senden kann. (Unter Windows: \myemph{rigctl.exe})
	\item \textbf{rotctl:} Ein Kommandozeilenprogramm, mit welchem man Befehle über die Kommandozeile an die Antennenrotoren senden kann. (Unter Windows: \myemph{rotctl.exe})
	\item \textbf{rigctld:} Ein Kommandozeilenprogramm, mit welchem man Befehle über das TCP/IP-Protokoll an das Funkgerät senden kann (Unter Windows: \myemph{rigctld.exe})
	\item \textbf{rotctld:}  Ein Kommandozeilenprogramm, mit welchem man Befehle über das TCP/IP-Protokoll an die Antennenrotoren senden kann (Unter Windows: \myemph{rigctld.exe})
\end{itemize}

\clearpage

Hierbei steht ,,\myemph{rot}'' für ,,Rotator'' (deutsch: Rotor), ,,\myemph{rig}'' für  ,,Rig'' (deutsch: Amateurfunkgerät) und das ,,\myemph{d} am Ende von \myemph{rigctld} und \myemph{rotctld} für ,,Deamon'' (deutsch: Hintergrundprozess).

\subsection{Verwendung}

Um die Verwendung dieser Kommandozeilenprogramme zu vereinfachen und um gleichzeitig die notwendige Konfiguration festzuhalten, wurde für die Programme \myemph{rigctld} und \myemph{rotctld} jeweils ein Batch-Skript erstellt (siehe Anhang \ref{chap:rigctldbat} und \ref{chap:rotctldbat}). Diese Skripte müssen beide gestartet werden, bevor unter GPredict eine Kommunikation mit den Rotoren bzw. mit dem Funkgerät stattfinden kann. 

\section{Inbetriebnahme unter Windows}

\begin{itemize}
	\parskip0pt
	\item \textbf{Batch-Skripte -> Hinweis auf Dummy-Skripte?!}
	\item \textbf{Interfaces unter GPredict-Einstellungen}
	\item \textbf{Konfiguration von rigctld und rotctld}
	\item \textbf{Hinweis darauf, dass die Skripte auf dem PC, der mit Funkgerät und Rotoren kommuniziert, ausgeführt werden müssen. GPredict kann auch von anderem PC ausgeführt werden und über das Netzwerk mit HamLib kommunizieren -> Einstellung der IP-Adresse und des Ports}
	\item \textbf{Blockschaltbild von (Hard- und) Softwarekomponenten}
\end{itemize}



% \section{Inbetriebnahme unter Linux}