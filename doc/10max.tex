%%%%%%%%%%%%%%%%%%%%%%%%%%%%%%%%%%%%%%%%%%%%%%%%%%%%%%%%%%%%%%%%%%%%%%%%%%%%%%%%%%
%%																				%%
%% File name: 		10max.tex													%%
%% Project name:	Hochleistungsantenne										%%
%% Type of work:	T3X00 project work											%%
%% Author:			Sarah Brückner, Maximilian Stiefel, Hannes Bohnengel		%%
%% Date:			27th Arpil 2016												%%
%% University:		DHBW Ravensburg Campus Friedrichshafen						%%
%% Comments:		Created in gedit with tab width = 4							%%
%%																				%%
%%%%%%%%%%%%%%%%%%%%%%%%%%%%%%%%%%%%%%%%%%%%%%%%%%%%%%%%%%%%%%%%%%%%%%%%%%%%%%%%%%

\chapter{Hintergründe}

\section{Bahnmechanik}
Es sei zu Beginn diese Absatzes darauf hingewiesen
\subsection{Die Keplerschen Gesetze}
\label{subsec:kepler}
Seit der Antike galt die Erklärung der Bewegung der Planeten und die Vorhersage dieser als eine große Herausforderung. Theorien von Ptolemaios mit seinem geozentrischen Weltbild und Kopernikus mit seinem heliozentrischen Weltbild führten bereits im 16. Jahrhundert zu brauchbaren Modellen zur Vorhersage der Planetenbewegungen. Diese Modelle unterlagen jedoch Ungenauigkeiten, „die in mit Instrumenten des 16. Jahrhunderts bereits messbaren Breichen lagen“ (siehe S. 20 in \cite{Raumflugm}). 
%%%%%%%%%%%%%%%%%%%%%%%%%%%%%%%%%%%%%%%%%%%%%%%%%%%%%%%%%%%%%%%%%%%%%%%%%%%%%%%%%%%%%%%%%%%%%%
\begin{figure}[h]                                                                           %%
	\centering                                                                            	%%
	\includegraphics[width=0.3\textwidth]{./images/kepler.jpg}                              %%
	\caption[Bahnelemente]{Johannes Kepler (1571-1630), Quelle: \cite{Wiki:Kepler}}         %%
	\label{fig:bahnelemente}                                                                %%
\end{figure}                                                                              	%%
%%%%%%%%%%%%%%%%%%%%%%%%%%%%%%%%%%%%%%%%%%%%%%%%%%%%%%%%%%%%%%%%%%%%%%%%%%%%%%%%%%%%%%%%%%%%%% 
Der mathematische Aufwand hinter diesen Modellen war enorm. Selbst das kopernikanische Weltbild, dass einige Vereinfachnugen mit sich brachte, bediente sich der Überlagerung einer Vielzahl von Kreisbwegungen, um das Verhalten der Planeten zu erklären. Resignierend zog sich zu der Zeit die katholische Kirsche und mit ihr viele Gelehrte auf den Standpunkt zurück, dass „die Frage, welche der Theorien die korrekte sei, [...] schlicht unbeantwortbar“ wäre (siehe S. 21 in \cite{Raumflugm}). 
\newpar
Ein deutscher Mathematiker und Astronom, Johannes Kepler, war hier anderer Auffassung. Er war überzeugter Kopernikaniker und stand im Dienste des Kaisers Rudolph II. Schließlich gelang es ihm aus seinen Beobachtungen drei einfache Gesetze herzuleiten. Seine Gesetze führten zu Vorhersagen der Planetenbewegungen nie da gewesener Präzision, welche er seinem Dienstherr widmend in den Rudolphinischen Tabellen niederschrieb. Steiner und Schlagerl schreiben in Ihrem Buch „Raumflugmechanik“, dass ohne die Vorarbeit Keplers keine Weltraumtechnik je existiert hätte (vgl. S. 21 in \cite{Raumflugm}). Die drei Gesetze lauten:
\begin{enumerate}
	\item Keplersches Gesetz: Die Planeten umlaufen die Sonne auf elliptischen Bahnen. In einem der Brennpunkte dieser Ellipsen befindet sich die Sonne. 
	\item Keplersches Gesetz: Die Linie von der Sonne zu einem Planeten überstreicht in gleichen Zeiten gleiche Flächen.
	\item Keplersches Gesetz: Die Quadrate der Umlaufzeiten zweier Planeten verhalten sich zueinander so wie die Kuben der großen Halbachsen ihrer Bahnellipsen. 
\end{enumerate}   
Kepler starb 1630 und damit 12 Jahre vor Isaac Newtons (1642-1726) Geburt. Mit seinen Werken hinterließ Kepler Newton alles, um das Gravitationsgesetz später herleiten zu können. 
\subsubsection{Das erste Keplersche Gesetz}
Durch die Annahme Planeten bewegen sich auf Ellipsen anstatt auf Kreisen, brach Kepler ein tausende Jahre altes Paradigma. Das mit der Ellipse war allerdings nicht seine Idee. Bereits im 11. Jahrhundert nahm ein arabischer Gelehrter namens Al-Zarkali (1029-1087) eine elliptische Bahn an, um die unregelmäßige Bewegung des Merkurs erklären zu können. Kepler kannte diese Idee durch die Lehren des Mathematikers und Astronomen Peuerbach (1423-1461), welcher die Ellipsen-Theorie im Abendland verbreitete.  
\newpar
Zunächst soll die Ellipse an sich betrachtet werden. Die einfachste Möglichkeit eine Ellipse zu konstruieren besteht darin zwei Nägel in einer Holzplatte mit einem Stück Schnur mit einer Schlaufe zu verbinden. Das Stück Schnur muss länger sein als der Abstand zwischen beiden Nägeln. Nimmt man nun einen Bleistift und drückt ihn in der Schlaufe gegen die Schnur, kann man die beiden Nägel mit Kontakt der Bleistiftspitze zum Holzbrett umrunden. Hält man die Schnur konstant auf Spannung, so ergibt sich eine Ellipse. Darüber hinaus muss der Punkt auf welchem die Schlaufe am Bleistift anliegt höher sein, als der Abschluss der Nagelköpfe. Im übertragenden Sinne beschreibt die folgende Mengendefinition dieses Experiment mit Bezug zu Abbildung \ref{fig:ellipse}. 
\begin{equation}
E = \{P | \overline{F_{1}P} + \overline{F_{2}P} = 2a = konstant\}
\end{equation}
\ensuremath{F_{1}} und \ensuremath{F_{2}} heißen Brennpunkte der Ellipse. \ensuremath{M} ist der Mittelpunkt der Ellipse. \ensuremath{S_{1}} und \ensuremath{S_{2}} sind die Haupt-, \ensuremath{S_{3}} und \ensuremath{S_{4}} die Nebenscheitel.      
%%%%%%%%%%%%%%%%%%%%%%%%%%%%%%%%%%%%%%%%%%%%%%%%%%%%%%%%%%%%%%%%%%%%%%%%%%%%%%%%%%%%%%%%%%%%%%
\begin{figure}[h]                                                                           %%
	\centering                                                                            	%%
	\includegraphics[width=0.6\textwidth]{./images/ellipse_new.jpg}                         %%
	\caption[Ellipse]{Ellipse, Quelle: Wikipedia zusätzlich eigener Überarbeitung}          %%
	\label{fig:ellipse}                                                                     %%
\end{figure}                                                                              	%%
%%%%%%%%%%%%%%%%%%%%%%%%%%%%%%%%%%%%%%%%%%%%%%%%%%%%%%%%%%%%%%%%%%%%%%%%%%%%%%%%%%%%%%%%%%%%%%      
Die Strecke \ensuremath{\overline{MS_{1}}} ist gleich der Strecke \ensuremath{\overline{MS_{2}}}. Man spricht bei der Länge dieser Strecke von der \textbf{großen Halbachse a}. Beide Strecken ergeben zusammen die Hauptachse \ensuremath{\overline{S_{1}S_{2}}}. Analog gibt es hierzu die Nebenachse, welche durch die Strecke \ensuremath{\overline{S_{3}S_{4}}} bestimmt wird. Die \textbf{kleinen Halbachsen} sind \ensuremath{\overline{MS_{3}}} und \ensuremath{\overline{MS_{4}}}. Diese haben die Längen \textbf{b}. Eine Ellipse kann auch als Stauchung eines Kreises mit dem Faktor \ensuremath{\frac{b}{a}} angesehen werden. 
\newpar
Die numerische Exzentrizität e' ist ein Maß für die Schlankheit der Ellipse. Sie ist definiert als
\begin{equation}
	e'=\frac{e}{a}
\end{equation}
Je größer die lineare Exzentrizität e im Verhältnis zu der großen Halbachse a wird, desto schlanker wird die Ellipse, da die Brennpunkte weiter vom Mittelpunkt entfernt sind. Das Wort numerisch gibt bei der Exzentrizität e' an, dass diese sich auf eine andere Größe (die große Halbachse) bezieht. Für eine Ellipse gilt \ensuremath{0 < e' < 1}. Für den Fall \ensuremath{e'=e=0} hat die Ellipse die selbe Erscheinung wie ein Kreis, da die Brennpunkte \ensuremath{F_1} und \ensuremath{F_2} im Mittelpunkt \ensuremath{M} liegen. Für \ensuremath{e'=1} entartet die Ellipse zu einer Geraden, da die kleine Halbachse b zu 0 wird. Um das zu zeigen wird die obige Gleichung noch mal herangezogen.
\begin{equation}
	e'^2=\frac{e^2}{a^2}=\frac{a^2-b^2}{a^2}=1-\left(\frac{b}{a}\right)^2 
	\label{equation_kepler_b}
\end{equation} 
\\Des Weiteren besitzt jede Ellipse einen Halbparameter \ensuremath{p}. Geht man davon aus, dass es einen Abstand \ensuremath{p'} gibt, welcher \ensuremath{p} bis zu einer die Ellipse umschließende Kreislinie verlängert, so gelten folgende Gleichungen
\begin{equation}
	\frac{p}{p'}=\frac{b}{a}
	\label{equation_kepler_p}
\end{equation}
Mit dem Satz eines alten Freudes folgt
\begin{equation}
p' = \sqrt{a^2-e^2}
\end{equation}
Jetzt ist klar, dass gilt
\begin{equation}
p=\frac{b}{a} \cdot p'= \frac{b}{a} \sqrt{a^2-e^2} = \frac{b^2}{a}
\label{equation_kepler_simple_p}
\end{equation}

Was nun noch fehlt ist „eine Gleichung, also eine analytische Beschreibung der Punkte einer Ellipse“ (siehe S.22 in \cite{Raumflugm}). Eine solche Gleichung ergibt sich mit dem Schnitt einer Ebene mit einem Kegel. 
%%%%%%%%%%%%%%%%%%%%%%%%%%%%%%%%%%%%%%%%%%%%%%%%%%%%%%%%%%%%%%%%%%%%%%%%%%%%%%%%%%%%%%%%%%%%%%
\begin{figure}[h]                                                                           %%
	\centering                                                                            	%%
	\includegraphics[width=0.8\textwidth]{./images/keplers_law1.jpg}                        %%
	\caption[Kegelschnitt]{Kegelschnitt, Quelle: S.24 in \cite{Raumflugm}}                  %%
	\label{fig:kegelsch}                                                                    %%
\end{figure}                                                                              	%%
%%%%%%%%%%%%%%%%%%%%%%%%%%%%%%%%%%%%%%%%%%%%%%%%%%%%%%%%%%%%%%%%%%%%%%%%%%%%%%%%%%%%%%%%%%%%%%      
Die Neigung der Schnittebene zur Kegelgrundfläche sei \ensuremath{\alpha}. Der Öffnungswinkel des Kegels sei \ensuremath{\epsilon}. Jetzt passiert etwas, dass das räumliche Denkvermögen herausfordert. In den Kegel wird eine (Dandelinsche) Kugel eingeschrieben. Diese Kugel berühre die Ebene im Punkt \ensuremath{F} und tangiere den Kegelmantel entlang eines Breitenkreises. Es ist einzusehen, dass der Punkt \ensuremath{F} auf der Hauptachse der Ellipse liegt. Der Schnittpunkt der entstehenden Ellipse und der Normale zur Hauptachse im Punkt \ensuremath{F} ist der Punkt \ensuremath{Q}. \ensuremath{P} stellt einen beliebigen Punkt auf der Ellipse dar. Die Verbindungslinie zwischen \ensuremath{F} und \ensuremath{P} hat zur Hauptachse die Neigung \ensuremath{\theta}. Der Abstand zwischen \ensuremath{F} und \ensuremath{P} ist \ensuremath{r}. \ensuremath{s} und \ensuremath{s_0} sind die Abstände der Punkte P und Q zum Berührkreis der Kugel gemessen entlang des Kegelmantels. 
\newpar
Man sieht nun, dass die Größen \ensuremath{r}, \ensuremath{s} und \ensuremath{\theta} von P abhängig sind. Es interessiere nun die mathematische Funktion \ensuremath{r(\theta)}, welche die Ellipsenbahn beschreibe (vgl. S.23 in \cite{Raumflugm}). Betrachtet wird nun zunächst die zweidimensionale Zeichnung rechts oben in Abb. \ref{fig:kegelsch}. Mit ein bisschen Nachdenken sieht man, dass
\begin{equation}
	(s_0-s) \cdot cos(\epsilon) = r \cdot cos (\theta) \cdot cos(\alpha)
	\label{equation1_kepler}
\end{equation} 
Doch woher kommt der Ausdruck \ensuremath{r \cdot cos(\theta)}? Hierzu werfe man einen Blick auf die zweidimensionale Abbildung der Schnittfläche/Ellipse links unten im Bild. Dieses Bild setze man nun in Relation zum Bild darüber. Der Abstand \ensuremath{r \cdot cos(\theta)} lässt sich nun auf die Hauptachse der Ellipse projizieren. \ensuremath{r}, die Projektionslinie für P die Hauptachse und F bilden nun eine rechtwinkliges Dreieck aus. Der Rest ist Trigonometrie. 
\newpar
In der Darstellung rechts unten in Abb. \ref{fig:kegelsch} ist folgende Beziehung auffindbar. 
\begin{equation}
 R^2 + s^2 = R^2 + r^2
\end{equation} 
Das bedeutet, dass \ensuremath{s} durch \ensuremath{r} in Gleichung \ref{equation1_kepler} ersetzt werden kann. 
\begin{equation}
	(s_0-r) \cdot cos(\epsilon) = r \cdot cos (\theta) \cdot cos(\alpha)
\end{equation}
Ausmultiplizieren ergibt
\begin{equation}
s_0 cos(\epsilon) - r cos(\epsilon) = r cos (\theta) cos(\alpha)
\end{equation}
Sortieren führt zu 
\begin{equation}
s_0 cos(\epsilon) = r cos (\theta) cos(\alpha) + r cos(\epsilon)
\end{equation}
Ausklammern und auflösen bringt
\begin{equation}
	r = \frac{s_0 cos(\epsilon) }{cos (\theta) cos(\alpha) + cos(\epsilon)}
		\label{equation2_kepler}
\end{equation}
Die entstandene Gleichung \ref{equation2_kepler} kann nun noch durch die Zusammenhänge \ensuremath{p=s_0} (Halbparameter) und \ensuremath{e'=\frac{cos(\alpha)}{cos(\epsilon)}} vereinfacht werden (vgl. S. 24 in \cite{Raumflugm}). Hierzu dividiert man Gleichung \ref{equation2_kepler} durch \ensuremath{cos(\epsilon)}. 
\begin{equation}
r = \frac{s_0}{1+\frac{cos(\alpha)}{cos(\epsilon)}cos (\theta)} = \frac{p}{1 + e' cos(\theta)} 
\label{equation3_kepler}
\end{equation}
Fertig ist die mathematische Version des ersten Keplerschen Gesetzes. 

\subsubsection{Das zweite Keplersche Gesetz}
%%%%%%%%%%%%%%%%%%%%%%%%%%%%%%%%%%%%%%%%%%%%%%%%%%%%%%%%%%%%%%%%%%%%%%%%%%%%%%%%%%%%%%%%%%%%%%
\begin{figure}[h]                                                                           %%
	\centering                                                                            	%%
	\includegraphics[width=0.6\textwidth]{./images/keplers_law2.jpg}                        %%
	\caption[Kegelschnitt]{Kegelschnitt, Quelle: S.26 in \cite{Raumflugm}}                  %%
	\label{fig:kegelsch}                                                                    %%
\end{figure}                                                                              	%%
%%%%%%%%%%%%%%%%%%%%%%%%%%%%%%%%%%%%%%%%%%%%%%%%%%%%%%%%%%%%%%%%%%%%%%%%%%%%%%%%%%%%%%%%%%%%%%      
Gleichung \ref{equation3_kepler} liefert noch keine Aussage über die zeitliche Änderung von \ensuremath{r} und \ensuremath{\theta}. Um eine Aussage über die zeitliche Änderung dieser Variablen treffen zu können, kann Keplers zweites Gesetz herangezogen werden: Die Fläche, welche die Verbindungslinie zwischen Sonne und einem Planet überstreicht ist zeitlich konstant. Die Fläche \ensuremath{\Delta A}, die in einem Zeitintervall \ensuremath{\Delta t} durch strichen wird ist genau
\begin{equation}
	\Delta A = \frac{1}{2}\left| \vec{r} \times \dot{\vec{r}} \right|\Delta t + F(\Delta t^2)
\end{equation}   
Die beiden Vektoren \ensuremath{r} und \ensuremath{\dot{r} \Delta t} spannen eine Fläche auf, welche ein Parallelogramm beschreibt. Das Kreuzprodukt ergibt einen Vektor dessen Länge dieser Fläche entspricht. Die Hälfte davon ist die Fläche des Dreiecks, die gesucht wird. Der Ausdruck \ensuremath{\dot{r} \Delta t} ist dabei sehr ungenau und beschreibt eigentlich nur die Änderung des Vektors \ensuremath{r}. Aus diesem Grund kommt noch der Fehlerterm \ensuremath{F} hinzu, der die Krümmung der Ellipse berücksichtigt. Bezieht man sich im nächsten Schritt auf infinitesimale Elemente, die wirklich gegen Null gehen, so erreicht man die gewünschte Genauigkeit. Der Fehlerterm wird überflüssig.  
\begin{equation}
dA = \frac{1}{2}\left| \vec{r} \times \dot{\vec{r}} \right|d t 
\label{equation_dA_kepler}
\end{equation}  
Man führt nun eine für jeden Planeten individuelle Konstante \ensuremath{h} ein, da sich das Verhältnis \ensuremath{\frac{dA}{dt}} nicht verändern darf. 
\begin{equation}
2 dA = h \cdot dt 
\end{equation}  
Um das mathematische Äquivalent zu dem sprachlich formulierten zweiten Gesetz zu erhalten, soll wie beim ersten Gesetz eine Abhängigkeit zu \ensuremath{r} und \ensuremath{\theta} hergestellt werden. Zu diesem Zweck wird die Gleichung einer Koordinatentransformation in Zylinderkoordinaten unterworfen (vgl. S. 25 f. in \cite{Raumflugm}). Es gilt also 
\begin{center}
	\ensuremath{x= r\, cos (\theta)}, \ensuremath{y=r\, sin (\theta)} und \ensuremath{z=z}. 
\end{center}
Gemäß der Definition von Zylinderkoordinaten darf man jetzt den Vektor \ensuremath{\vec{r}} auch anders schreiben:
\begin{equation}
	\vec{r} = r\,\vec{e_{r}} + z\,\vec{e_z}
\end{equation}
wobei folgendes generell über Zylinderkoordinatensysteme bekannt ist:
\begin{center}
	\ensuremath{\vec{e_{r}}=\left(\begin{array}{c}cos(\theta)\\ sin(\theta)\\0\end{array}\right)}, \ensuremath{\vec{e_{\theta}}=\left(\begin{array}{c}-sin(\theta)\\ cos(\theta)\\0\end{array}\right)}  und \ensuremath{\vec{e_z}=\left(\begin{array}{c}0\\0\\1\end{array}\right)}
\end{center}
Die Vektoren stehen allesamt senkrecht aufeinander, was man sieht, wenn man das Skalarprodukt bildet. Das liegt daran, dass das Skalarprodukt über die Summe der Längen der Vektoren multipliziert mit dem Kosinus des Winkels den sie einschließen definiert wird, welcher bei \ensuremath{\frac{\pi}{2}} bekanntlich Null ist.Da auch die Ableitung des Vektors \ensuremath{\vec{r}} (Geschwindigkeit) gesucht ist beginnt man zu differenzieren. Man wende hier zunächst die Summenregel, dann auf den ersten Ausdruck noch die Produkt- und die Kettenregel, da \ensuremath{\vec{e_{r}}} von \ensuremath{\theta} abhängt und diese wiederum von \ensuremath{t}. Es folgt
\begin{equation}
	\dot{\vec{r}} = \dot{r}\,\vec{e_r} + r\,\dot{\vec{e_r}} + \dot{z}\,\vec{e_z}
	\label{equation4_kepler}
\end{equation}
Setzt man nun   
\begin{center}	
	\ensuremath{\dot{\vec{e_{r}}}=\dot{\theta}\,\left(\begin{array}{c}-sin(\theta)\\ cos(\theta)\\0\end{array}\right)=\dot{\theta}\,\vec{e_{\theta}}}
\end{center}
in Gleichung \ref{equation4_kepler} ein, so ergibt sich  
\begin{equation}
\dot{\vec{r}} = \dot{r}\,\vec{e_r} + r\,\dot{\theta}\,\vec{e_{\theta}} + \dot{z}\,\vec{e_z}
\label{equation5_kepler}
\end{equation}
Man wählt nun die z-Achse geschickt, so dass diese senkrecht auf der Trägerebene der Ellipse steht (vgl. S.26 in \cite{Raumflugm}). Durch diesen Schachzug gilt für die zu betrachtenden Gleichungen \ensuremath{z=0}. Jetzt fällt Gleichung \ref{equation_dA_kepler} in sich zusammen
\begin{equation}
\frac{dA}{dt} = \frac{1}{2}\left| \vec{r} \times \dot{\vec{r}} \right| = \frac{1}{2}\left| r\,\vec{e_r} \times \left(\dot{r}\,\vec{e_r} + r\,\dot{\theta}\,\vec{e_{\theta}}\right) \right| 
\end{equation}  
Durch die Bilinearität des Kreuzprodukts folgt
\begin{equation}
\frac{dA}{dt} = \frac{1}{2}\left| \dot{r}\,r\,\left(\vec{e_r} \times \vec{e_r}\right) + r\,r\,\dot{\theta}\,\left(\vec{e_r} \times \vec{e_{\theta}}\right) \right| 
\end{equation}
Die Tatsache, dass das Kreuzprodukt eines Vektors mit sich selbst den Nullvektor ergibt und dem Umstand, dass \ensuremath{\vec{e_r}\perp\vec{e_\theta}\perp\vec{e_z}} ist, führt zu
\begin{equation}
\frac{dA}{dt} = \frac{1}{2}\left|r^2\,\dot{\theta}\,\vec{e_z} \right| =  \frac{1}{2}\,r^2\,\dot{\theta}=h=konstant
\label{equation6_kepler}
\end{equation}

\subsubsection{Das dritte Keplersche Gesetz}
Auf sein drittes Gesetz soll Kepler besonders stolz gewesen sein (vgl. S.27 in \cite{Raumflugm}). Nach seinem Gesetz stehen die Halbachsen \ensuremath{a_1} und \ensuremath{a_2} und die Umlaufzeiten \ensuremath{T_1} und \ensuremath{T_2} in dem Zusammenhang
\begin{equation}
	\frac{T_1^2}{T_2^2}=\frac{a_1^3}{a_2^3}
	\label{equation7_kepler}
\end{equation} 
Die Umlaufzeit lässt sich nun leicht mit der Konstanten \ensuremath{h}, welche für das zweite Keplersche Gesetz eingeführt wurde ausdrücken. Es wird hierfür die gesamte Ellipsenfläche (\ensuremath{A_{Ellipse}=ab\pi}) und konsequenterweise dann auch die gesamte Umlaufzeit \ensuremath{T} angenommen. Eingesetzt in Gleichung \ref{equation6_kepler} ergibt sich
\begin{equation}
	2ab\pi=hT.
\end{equation} 
Daraus folgt
\begin{equation}
	T=\frac{2ab\pi}{h}.
\end{equation}
Eingesetzt in Gleichung \ref{equation7_kepler} ergibt sich 
\begin{equation}
		\frac{T_1^2}{T_2^2}=\frac{a_1^3}{a_2^3}=\frac{a_1^2b_1^2h_2^2}{a_2^2b_2^2h_1^2}
\end{equation}
Bringt man nun \ensuremath{a_1^2b_1} und \ensuremath{a_2^2b_2^2} einmal durch Division und einmal durch Multiplikation eins weiter nach links, so kann man die Beziehung aus Gleichung \ref{equation_kepler_simple_p} ausnutzen und schreiben
\begin{equation}
\frac{a_1b_2^2}{a_2b_1^2}=\frac{h_2^2}{h_1^2}
\end{equation}
Oben und unten multipliziert man jetzt noch jeweils mit \ensuremath{1/a_1} und \ensuremath{1/a_2}.
\begin{equation}
\frac{b_2^2/a_2}{b_1^2/a_1}=\frac{h_2^2}{h_1^2}=\frac{p_2}{p_1}
\end{equation}
Das Resultat ist ein „Zusammenhang zwischen den Halbparametern \ensuremath{p_1}, \ensuremath{p_2} und den Bewegungskonstatnten \ensuremath{h_1}, \ensuremath{h_2}“ (siehe S.27 in \cite{Raumflugm}). Diese Größen sind aus dem zweiten Keplerschen Gesetz bekannt. 

\subsection{Die Bahnelemente}
%%%%%%%%%%%%%%%%%%%%%%%%%%%%%%%%%%%%%%%%%%%%%%%%%%%%%%%%%%%%%%%%%%%%%%%%%%%%%%%%%%%%%%%%%%%%%%
\begin{figure}[h]                                                                           %%
	\centering                                                                            	%%
	\includegraphics[width=0.8\textwidth]{./images/bahnelemente.png}                        %%
	\caption[Bahnelemente]{Bahnelemente, Quelle: \cite{Wiki:Bahnel}}                        %%
	\label{fig:bahnelemente}                                                                %%
\end{figure}                                                                              	%%
%%%%%%%%%%%%%%%%%%%%%%%%%%%%%%%%%%%%%%%%%%%%%%%%%%%%%%%%%%%%%%%%%%%%%%%%%%%%%%%%%%%%%%%%%%%%%%   
Die Bahnelemente dienen der Beschreibung einer Bewegung eines Himmelskörpers auf einer Umlaufbahn (meist einer Ellipse). Dieser Körper unterliegt den Keplerschen Gesetzen. Wird die Bewegung eines Himmelskörpers durch äußere Einflüsse (z.B. Gravitationskraft der Sonne) nicht gestört, so kann sie durch sechs Größen beschrieben werden. Diese Größen sind die Bahnelemente. Zwei Bahnelemente beschreiben die \textbf{Form der Bahn}, drei legen die \textbf{Lage der Bahn} im dreidimensionalen Raum fest und ein Bahnelement gibt an zu welcher \textbf{Zeit} sich der Himmelskörper wo auf der Bahn befunden hat. 
\newpar
Diese Bahnelemente reichen in der Praxis nicht aus, um die Position eines Himmelskörpers z.B. eines Satelliten mit einem Vorhersagemodell berechnen zu können. Aus diesem Grund werden die Bahnelemente meist um von Vorhersagemodellen benötigten Informationen ergänzt.       
Im Folgenden werden die Bahnelemente in Ihrer Bedeutung anhand der Abbildung \ref{fig:bahnelemente} erläutert. 

\subsubsection{Gestalt der Bahn}
Um die Gestalt der Bahn zu beschreiben wird die \textbf{numerische Exzentrizität (1) \ensuremath{e'}} und die Angabe der Länge der \textbf{großen Halbachse (2) \ensuremath{a}} benötigt. Diese beiden Größen werden ausführlich im Absatz \ref{subsec:kepler} beschrieben. Es braucht nicht mehr, um eine elliptische Bahn zu definieren. Zur Erinnerung: Die große Halbachse \ensuremath{a} ist die Länge der Strecke vom Mittelpunkt zu den Hauptscheiteln. Die (lineare) Exzentrizität \ensuremath{e} gibt die Schlankeit der Ellipse an. Wobei die numerische Exzentrizität \ensuremath{e'} auf die große Halbachse normiert ist.  
\newpar
Die Abhängigkeit der Umlaufzeit eines Satelliten von den Bahndimensionen (v.a. Länge der großen Halbachse ) und die elliptische Bahnform sind der „quadratischen Abnahme der Gravitationsbeschleunigung [geschuldet]“ (siehe S. 87 in \cite{HandRaum}). 
\begin{equation}
\ddot{\vec{r}}=-\frac{GM_{Erde}}{r^2}\cdot\frac{\vec{r}}{r}
\end{equation}    
\ensuremath{GM_{Erde} \approx 398600,4\,\frac{km^3}{s^2}} ist dabei das konstante Produkt aus Gravitationskonstante und Masse der Erde. Das Stichwort lautet hier Impulserhaltungssatz. In einem geschlossenen System kann Impuls weder vernichtet noch erschaffen werden. Bis zum Perigäumsdurchgang nimmt ein Satellit Impuls von der Erde über die Kraft, die zum Erdmittelpunkt hin wirkt auf. Danach wird Impuls an dier Erde abgegeben \ensuremath{\rightarrow} der Satellit verlangsamt sich. Nach dem Apogäumsdurchgang fängt der Satellit wieder an schneller zu werden. Er nimmt wieder Impuls auf.  
  
\subsubsection{Lage der Bahn}
Es erscheint sinnvoll, dass die Lage im Raum als nächstes festgelegt wird. Auch scheint es sinnig, dass man drei Bahnellemente benötigt, um die Lage der Bahn im dreidimensionalen Raum festzulegen. 
\newpar
Man stelle sich nun vor, dass die Ellipse, auf welcher der Satellit sich bewegt, in einer Ebene liege.  Diese Ebene schneide man mit einer Referenzebene (z.B. Schnittebene durch den Erdäquator). Die \textbf{Inklination (3) \ensuremath{i}} ist nun ein Winkel, „um den die Bahnebene gegenüber der [Referenzebene] geneigt ist“ (siehe S.88 in \cite{HandRaum}). Um die Nebenachse der Ellipse herum kann durch diese Festlegung keine Rotation mehr stattfinden. Die Ellipse, die ja in der Ebene liegt, kann sich jetzt noch in dieser Ebene um eine zur Ebene orthogonalen Achse drehen. Um das zu unterbinden wird ein weiterer Winkel, das \textbf{Argument des Perigäums (4) \ensuremath{\omega}}, eingeführt. Das Argument des Perigäums ist der Winkel, den die Knotenlinie und die Perigäumsrichtung (Hauptachse) einschließen. Das Perigäum liegt am Ende der Hauptachse, auf welches sich der Satellit zu bewegt, nachdem er den aufsteigenden Knoten passiert hat. Der aufsteigende Knoten ist der Punkt in dem der Satellit die Referenzebene auf seiner elliptischen Umlaufbahn (von Süd nach Nord) durchstößt. Die Knotenlinie ist die Schnittgerade der Ellipse mit der Referenzebene. Darüber hinaus gibt es das Apogäum, welches nach dem absteigenden Knoten durchlaufen wird, der am anderen Ende der Nebenachse beheimatet ist. Eine andere Definition für das Perigäum ist der Punkt des geringsten Abstands des Satelliten zur Erde. Analog lässt sich das Apogäum als Punkt des größten Abstands des Satelliten zur Erde festlegen. Die Ellipse dreht sich nun nicht mehr in der Ebene. Eine Sache wurde jetzt noch nicht betrachtet. Die Ebene kann sich immer noch entlang Ihrer Hauptachse drehen. Die \textbf{Rektaszension des aufsteigenden Knotens (5) \ensuremath{\Omega}} gibt nun den Winkel an, den die Knotenlinie und der Vektor vom Ellipsenmittelpunkt zum Frühlingspunkt \ensuremath{\gamma} einschließen. Doch was ist der Frühlingspunkt? „Als Frühlingspunkt (auch Widderpunkt) wird in der Astronomie der Schnittpunkt des Himmelsäquators mit der Ekliptik bezeichnet, an dem die Sonne zum Frühlingsanfang der Nordhalbkugel (= Herbstanfang der Südhalbkugel) steht“ (siehe \cite{Wiki:Frue}). Die Derehung um die Hauptachse ist jetzt also auch nicht mehr möglich. Der Bahnlage wurden somit alle drei Freiheitsgrade genommen.     
\newpar
Die Bewegung eines Satelliten erfolgt in einer nicht veränderbaren Bahnebene mit dem Bahnmittelpunkt gleich dem Erdmittelpunkt. Dies impilziert das erste Keplersche Gesetz. Der Grund hierfür ist „dass die Anziehung der Erde (in erster Näherung) immer zum Erdmittelpunkt gerichtet ist“ (siehe S. 87 in \cite{HandRaum}). Zu keinem Zeitpunkt gibt es somit einen Kraftvektor, der Senkrecht zum Ortsvektor \ensuremath{\vec{r}} wirkt und somit die Bahnebene verschieben könnte (Annahme: Mittelpunkt des Koordinatensystems ist der Erdmittelpunkt). Daraus resultiert wiederum, dass eine einmal eingeschlagene Bahnebene ohne eine äußere Kraft nicht wieder verlassen werden kann.
 
\subsubsection{Zeitlicher Bezug und Kepler-Gleichung}
Bis jetzt ist alles recht schön und anschaulich. Es ist klar wo die Bahnebene im Raum ist und wie diese aussieht. Wir haben durch das zweite Keplersche Gesetz eine Gleichung, die uns die Position des Satelliten in Abhängigkeit des Winkels \ensuremath{\theta} verrät. Diese ist 
\begin{equation}
	r = \frac{p}{1 + e' cos(\theta)} 
\end{equation}
\ensuremath{\theta} ist der Winkel, den die Verbindungslinie zwischen einem Brennpunkt und dem Satellit mit der Hauptachse einschließt. Es gibt allerdings noch ein Haar in der Suppe. Aufgrund des Flächensatzes ist die Winkelgeschwindigkeit des Satelliten nicht konstant (bzw. nur für eine in der Natur nicht vorkommende Kreisbahn konstant). \ensuremath{\theta} ist eine Funktion von \ensuremath{t}. Die sog. wahre Anomalie \ensuremath{\theta(t)} schwankt um die mittlere Anomalie \ensuremath{M} (nicht zu verwechseln mit der Erdmasse). Man stößt nun auf das Zweikörperproblem (Keplerproblem). Dies ist die Angabe der wahren Anomalie \ensuremath{\theta(t=t_x)} zu einem vorgegebenen Zeitpunkt \ensuremath{t_x}.  Wie kommt die mittlere Anomalie nun zustande? Zunächst sollte ein genauer Blick auf folgende Abbildung \ref{fig:kepler_gl} geworfen werden. Die Abbildung stellt eine Momentaufnahme einer Situation dar.     
%%%%%%%%%%%%%%%%%%%%%%%%%%%%%%%%%%%%%%%%%%%%%%%%%%%%%%%%%%%%%%%%%%%%%%%%%%%%%%%%%%%%%%%%%%%%%%
\begin{figure}[h]                                                                           %%
	\centering                                                                            	%%
	\includegraphics[width=0.4\textwidth]{./images/keplers_equation.jpg}                    %%
	\caption[Kepler-Gleichung]{Kepler-Gleichung, Quelle: \cite{Wiki:KeplerGl}, 
								Grafik ist eigens modifiziert}              				%%
	\label{fig:kepler_gl}                                                                	%%
\end{figure}                                                                              	%%
%%%%%%%%%%%%%%%%%%%%%%%%%%%%%%%%%%%%%%%%%%%%%%%%%%%%%%%%%%%%%%%%%%%%%%%%%%%%%%%%%%%%%%%%%%%%%%   
Wie schon bei früheren Betrachtungen wird die Ellipse (Orbit) in einen Umkreis eingeschrieben. Ein fiktiver (mittlerer) Satellit \ensuremath{Y} laufe entlang des Umkreises um den Punkt \ensuremath{C} (Mittelpunkt der Ellipse und des Kreises, kein Ellipsenbrennpunkt). Der wahre Satellit \ensuremath{P} bewege sich entlang der Ellipse. Die Lage des fiktiven Satelliten \ensuremath{Y} wird zum Zeitpunkt \ensuremath{t} mittels des Winkels \ensuremath{M} dargestellt. Hierbei wird unterstellt, dass der Satellit zum Zeitpunkt \ensuremath{t_P} im Perigäum (oft auch Periapsis genannt) \ensuremath{M=0} war. Es gilt nun
\begin{equation}
	M=2\pi\,\frac{t-t_P}{U}=n\,(t-t_P)
\end{equation}   
Hierbei ist \ensuremath{n} die Winkelgeschwindigkeit. \ensuremath{S} ist der Erdmittelpunkt (Schwerezentrum). Um nun die Kurve zu der Ellipse und dem tatsächlichen Satellit zu bekommen wird das zweite Keplersche Gesetz für diesen konkreten Fall herangezogen. In gleichen Zeiträumen werden gleiche Flächen von den Fahrstrählen der Satelliten unabhängig von der Bahnform (egal ob elliptisch oder kreisförmig mit \ensuremath{e'=e=0}) überstrichen. Hier spricht man von einer bijektiven Abbildung der Ellipsen- und Kreisflächensegmente. Letztere sind leicht zu berechnen. Es wird also praktisch ein Umweg über den Umkreis gegangen. Das zweite Keplersche Gesetz angewandt auf den vorliegenden Fall, führt bezüglich der Zeitspanne einer Überstreichung der gesamten Fläche zu
\begin{equation}
	\frac{A_{Kreis}}{A_{Ellipse}}=\frac{\pi\,a^2}{\pi\,a\,b}
\end{equation}
Nach dem zweiten Keplerschen Gesetz ist dieses Vehätlnis dasselbe für eine infinitesimale Fläche. So gilt
\begin{equation}
	\frac{A_{CYZ}}{A_{SPZ}}=\frac{A_{Kreis}}{A_{Ellipse}}=\frac{a}{b}
	\label{equation1_kepler_equ}
\end{equation}
\newpar
Kepler führte als eine Hilfsgröße die exzentrische Anomalie \ensuremath{E} ein. Hierzu wird \ensuremath{P} auf den Hilfskreis projeziert. Es entsteht der Punkt \ensuremath{X}. Der Winkel zwischen Hauptachse und \ensuremath{X} ist die exzentrische Anomalie. Durch den in Gleichung \ref{equation1_kepler_equ} hergeleiteten Zusammenhang kann folgende Aussage getroffen werden. 
\begin{equation}
	A_{SXZ}=\frac{a}{b}A_{SPZ}	
	\label{equation2_kepler_equ}
\end{equation}  
Setzt man Gleichung \ref{equation1_kepler_equ} in Gleichung \ref{equation2_kepler_equ} ein, so ergibt sich
\begin{equation}
	A_{SXZ}=A_{CYZ}	
	\label{equation3_kepler_equ}
\end{equation}
Mit der Gleichung \ref{equation3_kepler_equ} ist ein indirekter Zusammenhang zwischen der mittleren Anomalie (Punkt \ensuremath{Y}) und der exzentrischen Anomalie (Punkt \ensuremath{X}) gefunden. Der Fahrstrahl \ensuremath{\overline{CY}} in der Zeit einer Periode \ensuremath{U} eine Fläche von \ensuremath{\pi\,a^2} überstreicht, so überstreicht er in einer Zeit \ensuremath{t} eine Fläche, die um den Faktor \ensuremath{\frac{M}{2\,\pi}} kleiner ist. Es folgt 
\begin{equation}
	A_{CYZ}=\frac{M}{2\,\pi}\,\pi\,a^2=\frac{a^2}{2}M	
	\label{equation4_kepler_equ}
\end{equation}  
Analog gilt dies für den Fahrstrahl \ensuremath{\overline{CX}}.
\begin{equation}
	A_{CXZ}=\frac{a^2}{2}E	
	\label{equation5_kepler_equ}
\end{equation}
Weiter geht's mit noch mehr Geometrie. Die Fläche \ensuremath{A_{CXZ}} kann in zwei Teilflächen zerlegt werden.
\begin{equation}
	A_{CXZ}=A_{CSX}+A_{SXZ}
	\label{equation6_kepler_equ}
\end{equation}
Die Fläche \ensuremath{A_{CSX}} kann als die Fläche eines Dreiecks aufgefasst werden. Dabei ist \ensuremath{a\cdot e'} die Basis und \ensuremath{a\,sin(E)} die Höhe des Dreiecks. 
\begin{equation}
	A_{CSX}=\frac{1}{2}\cdot a\cdot e' \cdot a\,sin(E)=\frac{a^2}{2}\,e'\,sin(E)
	\label{equation7_kepler_equ}
\end{equation}
Als letzten Schritt setze man nun noch Gleichung \ref{equation3_kepler_equ} (unter Beachtung von Gleichung \ref{equation4_kepler_equ}), Gleichung \ref{equation5_kepler_equ} und Gleichung \ref{equation7_kepler_equ} in Gleichung \ref{equation6_kepler_equ} ein.
\begin{equation}
	\frac{a^2}{2}E=\frac{a^2}{2}\,e'\,sin(E)+\frac{a^2}{2}M
	\label{equation8_kepler_equ}
\end{equation}
Mit ein paar Tricks aus der Grundschule entsteht die \textbf{Kepler-Gleichung} wie sie im Buche steht.
\begin{equation}
	E-e'\,sin(E)=M=U\cdot t
	\label{equation_kepler_equ_final}
\end{equation}
Diese Gleichung kann nicht nach \ensuremath{E} aufgelöst werden und so keine schöne Funktion \ensuremath{E(t)} gewonnen werden. Doch Gott sei dank, schenke er Newton die Eingebung des Newton-Verfahrens zur Nullstellensuche. Das Newton-Verfahren ist im Grunde eine rekursive Folge, wobei jedes Folgeglied allgemein mit
\begin{equation}
	x_{n+1}=x_n-\frac{f(x_n)}{f'^{x}(x_n)}
	\label{equation9_kepler_equ}
\end{equation}
zu bestimmen ist. Man stellt also Gleichung \ref{equation_kepler_equ_final} entsprechend um und leite ab.
\begin{equation}
	f(E)=E-e'\,sin(E)-M \stackrel{!}{=} 0
	\label{equation10_kepler_equ}
\end{equation}
\begin{equation}
	f'^{E}(E)=1-e'\,cos(E) 
	\label{equation11_kepler_equ}
\end{equation}
Es folgt die rekursive Folge für die Kepler-Gleichung. 
\begin{equation}
	E_{n+1}=E_n-\frac{E-e'\,sin(E)-M}{1-e'\,cos(E) }
	\label{equation12_kepler_equ}
\end{equation}
Als Startwert wird in \cite{HandRaum} \ensuremath{E_0=M} empfohlen. Die rekursive Folge Verhält sich dabei so, dass sie gegen einen entsprechenden Wert für die exzentrische Anomalie \ensuremath{E} konvergiert. Wenn sich entsprechende Nachkommastellen nach gewisser Genauigkeit nicht mehr ändern, so hat man die Nullstelle gefunden. Mit diesem Wert für \ensuremath{E(t=t_x)} kann die wahre Anomalie \ensuremath{\theta(t=t_x)} bestimmt werden. Zu beachten ist, dass in die Kepler-Gleichung natürlich vor der Anwendung des Newton-Verfahrens auch der gewünschte Zeitpunkt \ensuremath{t_p} eingesetzt wird.     
\newpar
Wie bestimmt man nun die wahre Anomalie, also den Winkel, der zu dem Zeitpunkt gehört aus welchem wir dann wiederum einen Vektor \ensuremath{r} gewinnen können? Hierzu werden die Größen \ensuremath{x'} und \ensuremath{y'} in Abbildung \ref{fig:kepler_gl} herangezogen. Man sucht nun nach einer Beziehung zwischen diesen Größen und bekannten Größen aus dem Fundus der Bahnelemente die Form der Bahn betreffend (dazu gehören natürlich auch abgeleitete Größen). 
Es ist nachvollziehbar, dass 
\begin{equation}
	x'=r\,cos(\theta)=cos(E)\,a-e=a\left(cos(E)-e'\right)
\end{equation}
ist. Für \ensuremath{y'} muss man etwas tiefer in die Trickkiste greifen. Man geht aus von Gleichung \ref{equation_kepler_p}. Diese stelle man nach \ensuremath{p} um.
\begin{equation}
	p=p'\,\frac{b}{a}
\end{equation} 
Für \ensuremath{p'} kann der Term \ensuremath{sin(E)\,a} eingesetzt werden. Mit dem Satz des Pythagoras kann \ensuremath{b} auch mit den größen \ensuremath{a} und \ensuremath{e} ausgedrückt werden. Es folgt  
\begin{equation}
	p=y'=sin(E)\,a\,\frac{\sqrt{a^2-e^2}}{a}=sin(E)\,a\,\sqrt{1-e'^2}
\end{equation} 
Aus diesen beiden Größen kann unmittelbar die wahre Anomalie \ensuremath{\theta} zu einem gegebenen Zeitpunkt bestimmt werden.
\begin{equation}
	\theta=arctan\left(\frac{y'}{x'}\right)=arctan\left(\frac{sin(E)\,a\,\sqrt{1-e'^2}}{a\left(cos(E)-e'\right)}\right)=arctan\left(\frac{sin(E)\,\sqrt{1-e'^2}}{cos(E)-e'}\right)
\end{equation}
\newpar
Nun ist also der zeitliche Bezug bekannt. Hier nochmal eine kleine Darstellung wie vorzugehen ist.
\begin{center}
	\smartdiagram[circular diagram:clockwise]{\ensuremath{t} gegeben, \ensuremath{M=n\cdot (t-t_P)} bestimmen, \ensuremath{E} bestimmen, \ensuremath{x'} und \ensuremath{y'} bestimmen, \ensuremath{\theta} und \ensuremath{r} bestimmen} 
\end{center}
Führt man den Vorgang wie oben gezeigt iterativ aus, so erhält man einen zweidimensionalen Vektor \ensuremath{\vec{r}}. Wenn man nun scharf nachdenkt, so fällt einem auf, dass man nochmal eine Information benötigt. Es braucht einen Zeitpunkt \ensuremath{t_P}. Dies muss ein Zeitpunkt sein, bei welchem der Satelliten im Perigäum stand. Als Analogie könnte man die Phase eines Signals in der Elektrotechnik angeben. Genau diesen Zeitpunkt liefern die Bahnelemente als zeitlichen Bezug.
\newpar
Hier nochmal eine Übersicht aller Bahnelemente. 
\begin{itemize}
	\item Gestalt der Bahn
	\begin{itemize}
		\item die große Halbachse \ensuremath{a}
		\item die numerische Exzentrizität \ensuremath{e'}
	\end{itemize}
	\item Lage der Bahn
	\begin{itemize}
		\item die Inklination \ensuremath{i}
		\item die Rektaszension des aufsteigenden Knotens \ensuremath{\Omega}
		\item das Argument des Perigäums \ensuremath{\omega}
	\end{itemize}
	\item Zeitlicher Bezug
	\begin{itemize}
		\item die Epoche (wann ist der Sattelit im Perigäum gewesen) \ensuremath{t_P}
	\end{itemize}
\end{itemize} 

\subsection{Von den Bahnelementen zum Vektor}
Im nächsten Schritt trägt man alle Informationen aus den Bahnelementen zusammen, um einen Vektor \ensuremath{\vec{r}} im dreidimensionalen kartesischen Koordinatensystem zu erhalten. Der Ursprung des Koordinatensystems soll im Erdzentrum liegen und der Vektor \ensuremath{\vec{r}} soll ein Ortsvektor des Satelliten sein. Die x-y-Ebene dieses Koordinatensystems ist der Äquator. Das Ergebnis der bisherigen Berechnungen ist der Vektor
\begin{equation}
	\vec{r'}=a\cdot\left(\begin{array}{c}cos(E)-e'\\ sin(E)\,\sqrt{1-e'^2} \\0\end{array}\right)
\end{equation}
Die Satellitenbahn liegt nach der obigen Gleichung in der äquatorialen Ebene. Der Erdmittelpunkt liegt im Brennpunkt der Ellipse. Anstatt die Bahnebene zu drehen. Dreht man nun das Koordinatensystem. Allerdings in die mathematisch entgegengesetzte Richtung. Hierfür können natürlich nur die Winkel verwendet werden, welche in den Bahnelementen für die Lage der Bahn mitgeliefert werden. Bei der Drehung wird auf Drehmatrizen zurückgegriffen. Eine gute Veranschaulichung dieser Drehungen findet man \href{https://www.youtube.com/watch?v=QZrYaKwZwhI}{\textit{in dieser Animation}}. Der gesuchte Vektor ergibt sich nun zu
\begin{equation}
	\vec{r}=\mathbf{R_Z}(-\Omega)\,\mathbf{R_X}(-i)\,\mathbf{R_Z}(-\omega)\cdot a\cdot\left(\begin{array}{c}cos(E)-e'\\ sin(E)\,\sqrt{1-e'^2} \\0\end{array}\right) 
\end{equation}  
Hierbei stellen \ensuremath{\mathbf{R_X}} und \ensuremath{\mathbf{R_Z}} folgende Drehmatrizen dar (nachzuschlagen auf S.89 in \cite{HandRaum}). 
\begin{equation}
\mathbf{R_X}(\alpha) = 
	\left(
		\begin{array}{ccc}
			1 & 0 & 0 \\
			0 & +cos\,\alpha & +sin\,\alpha\\
			0 & -sin\,\alpha & +cos\,\alpha
		\end{array}
	\right)
\end{equation}
\begin{equation}
\mathbf{R_Z}(\alpha) = 
	\left(
		\begin{array}{ccc}
			+cos\,\alpha & +sin\,\alpha & 0\\
			-sin\,\alpha & +cos\,\alpha & 0\\
			0 & 0 & 1 \\
		\end{array}
	\right)
\end{equation}
Eine Frage, die sich aufdrängen mag ist: Warum kann die Drehung der Ellipse in der Bahnebene (Argument des Perigäums \ensuremath{\omega}) einfach auf eine Drehung der z-Achse zurüclgeführt werden? Man kann es sich auch so vorstellen, dass zunächst die Ellipse noch in der äquatorialen Ebene liegt, die z-Achse dann um \ensuremath{-\omega} gedreht wird, dann die x-Achse um \ensuremath{-i} und letztendlich die z-Achse nochmal um \ensuremath{-\Omega} gedreht wird.
\newpar
Nach \cite{HandRaum} ergibt sich nach der Ausführung der Drehungen der Vektor
\begin{equation}
\vec{r} = r\cdot  
	\left(
		\begin{array}{c}
				cos(\omega+\theta)\,cos(\Omega)-sin(\omega+\theta)\,cos(i)\,sin(\Omega)\\
				cos(\omega+\theta)\,sin(\Omega)+sin(\omega+\theta)\,cos(i)\,cos(\Omega)\\
				sin(\omega+\theta)\,sin(i)				
		\end{array}
	\right)
	\label{equation1_kep_elements}
\end{equation}
Durch Ableitung nach der Zeit kommt man wie in \cite{HandRaum} angegeben auf
\begin{equation}
	\vec{v}=\mathbf{R_Z}(-\Omega)\,\mathbf{R_X}(-i)\,\mathbf{R_Z}(-\omega)\cdot\frac{\sqrt{GM_{Erde}\,a}}{r}\cdot\left(\begin{array}{c}-sin(E)\\ cos(E)\,\sqrt{1-e'^2} \\0\end{array}\right) 
	\label{equation2_kep_elements}
\end{equation} 
Gleichung \ref{equation1_kep_elements} lässt sich so interpretieren, dass der Term \ensuremath{\sqrt{\frac{GM_{Erde}}{r^2\,a}}} die zeitliche Ableitung der exzentrischen Anomalie \ensuremath{E} sein muss, da die Kettenregel der Differentialrechnung greift und die Rotationsmatrizen keine zetilichen Abhängigkeiten aufweisen.
\newpar
Mit den Gleichungen \ref{equation1_kep_elements} und \ref{equation2_kep_elements} lässt sich nun die Antenne fast ausrichten und nachführen. Der nächste Schritt ist die Berücksichtigung der Position der Bodenstation auf der Erde. Um sich das Leben nicht unnötig schwer zu machen, verwendet geografische Koordinaten und gewinnt daraus einen Ortsvektor für das kartesische Koordinatensystem im Erdmittelpunkt. Es wird hierbei vernachlässigt, dass die Erde keine Kugel sondern annähernd ein Ellipsoid ist. Ganz korrekt würde man es machen, wenn man sich des World Geodetic System 1984 (WGS84) bedient. Laut \cite{Wiki:Geo} können durch diese Vernachlässigung Orstverschiebungen (der Bodenstation) von bis zu 20km auftreten. Es soll jedoch zunächst möglichst einfach verstanden werden, wie ein Programm wie GPredict vorgeht und auch zur Überprüfung der Daten, die GPredict liefert reicht die Genauigkeit aus, die mit dieser Vernachlässigung erreicht wird .
\newpar  
Im ersten Schritt zur  \textbf{Beschreibung der Bahn im erdfesten System} muss das Koordinatensystem nochmal an der z-Achse gedreht werden. Das liegt daran, dass die x-y-Ebene zwar im Äquator liegt, die x-Achse jedoch nicht auf den nullten Längengrad, sondern auf den Frühlingspunkt zeigt. Bei dieser Achsentransformation wird auch die Rotation der Erde mit berücksichtigt. Die x-Achse wird in den sog. Meridian von Greenwich gezogen. Die Drehung der z-Achse wird mit dem Winkel \ensuremath{\Theta} vollzogen. Dieser Winkel wird dabei als „Sternenzeit bezeichnet und häufig im Stundenmaß (1h entspricht 15\degree) ausgedrückt“ (siehe S. 89 in \cite{HandRaum}). Für diese einfacheren Angelegenheiten kann so in \cite{HandRaum} beschrieben die Sternenzeit nach folgender Gleichung eruiert werden.
\begin{equation}
	\Theta = 280,4606\degree+360,9856473\degree\cdot d
\end{equation} 
In dieser Gleichung stellt \ensuremath{d} die Tage seit dem 1. Januar 2000 dar. Mit gegebener UT1 kann so die genaue Sternzeit berechnet werden. Mit der Sternzeit ergibt sich nun die letzte Drehtransformation. 
\begin{equation}
	\vec{r}_{Greenwich}=\mathbf{R_Z}(\Theta) \cdot \vec{r}
\end{equation} 
Dadurch ist die erdfeste Position des Satelliten aus den inertialen Koordinaten bestimmt worden. Nach \cite{HandRaum} ergibt sich die erste zeitliche Ableitung zu
\begin{equation}
	\vec{v}_{Greenwich}=\mathbf{R_Z}(\Theta)\cdot\vec{v}-\left(\begin{array}{c}0\\0\\\omega_{Erde}\end{array}\right)\times\vec{r}_{Greenwich}
	\label{equation3_kep_elements}
\end{equation}   
Die Interpretation von Gleichung \ref{equation3_kep_elements} ist, dass sich die Geschwindigkeit \ensuremath{\vec{v}_{Greenwich}} im erdfesten System vom inertialen System durch einen Term unterscheidet, der von der Drehgeschwindigkeit \ensuremath{\omega_{Erde}=7,29212\cdot 10^{-5}\frac{rad}{s}} und dem Abstand von der Erde abhängt. 
\newpar
Liegt eine Bodenstation von der die Funkverbindung zum Satellit aufgebaut werden soll nicht zufällig auf dem Schnittpunkt zwischen dem nullten Längengrad und dem Äquator, so muss der Vektor \ensuremath{\vec{r}_{Greenwich}} selbstverständlich angepasst werden. Das grobe Vorgehen ist aus der analytischen Geometrie bekannt. Der Vektor \ensuremath{\vec{r}_{Greenwich}} stellt den Ortsvektor zum Satelliten dar. Der Vektor \ensuremath{\vec{r}_{Bodenstation}} stellt den Orstvektor zur Bodenstation dar. Um einen Vektor von der Bodenstation zum Satellit zu erhalten rechne man Spitze minus Anfang.
\begin{equation}
	\vec{r}_{Bodenstation,Satellit}=\vec{r}_{Satellit}-\vec{r}_{Bodenstation}
\end{equation} 
wobei 
\begin{center}
	\begin{math}
		\vec{r}_{Satellit}=\vec{r}_{Greenwich}
	\end{math}
\end{center}


%%%%%%%%%%%%%%%%%%%%%%%%%%%%%%%%%%%%%%%%%%%%%%%%%%%%%%%%%%%%%%%%%%%%%%%%%%%%%%%%%%%%%%%%%%%%%%
\begin{figure}[h]                                                                           %%
	\centering                                                                            	%%
	\includegraphics[width=0.4\textwidth]{./images/geographic_coordinates.jpg}              %%
	\caption[Geografische Koordinaten]{Geografische Koordinaten, Quelle: \cite{Wiki:Geo}}                        												%%
	\label{fig:geo}                                                                         %%
\end{figure}                                                                              	%%
%%%%%%%%%%%%%%%%%%%%%%%%%%%%%%%%%%%%%%%%%%%%%%%%%%%%%%%%%%%%%%%%%%%%%%%%%%%%%%%%%%%%%%%%%%%%%%
Man werfe zunächst einen Blick auf das dem populären Kugelkoordinatensystem nicht ganz unähnlichen geografische Koordinatensystem (Abbildung \ref{fig:geo}). Die Koordinaten der Bodenstation im geografischen Koordinatensystem lassen sich ganz leicht ohne vor Ort zu sein mittels \href{http://www.openstreetmap.org/search?query=47.66530%2C9.44805#map=19/47.66530/9.44805}{\textit{OpenStreetMap}}. Dabei entspricht der erste Wert dem Breiten- der zweite dem Längengrad. Der mittlere Erddurchmesser \ensuremath{d_{Erde}} beträg nach \cite{Wiki:Erde} in etwa \ensuremath{12.742 km}.
\newpar
Man stelle sich nun die Bodenstation als einen Punkt auf der Kugeloberfläche vor. Schneidet man entlang des Längengrades der Bodenstation durch die Kugel, so lässt sich mit einfacher Trigonomoetrie die Information des Breitengrades in das kartesische Koordinatensystem übetragen. Die z-Komponente des gesuchten Ortsvektors \ensuremath{\vec{r}_{Bodenstation}} ist der Sinus des Breitengrades multipliziert mit dem halben Erddurchmessser. Die Ankatehte des Dreiecks, dass sich jetzt ausgebildet hat spielt bei der Betrachtung von x- und y-Komponente eine wichtige Rolle. Diese Ankathete hat die Länge des Kosinus des Breitengrads multipliziert mit dem halben Erddruchmesser. Um die x- und die y-Komponente zu ermitteln schneidet man entlang des Äquators durch die Erdkugel hindurch und wirft einen Blick auf die Schnittfläche. Die eben betonte Ankathete definiert mit der x-Achse ein weiteres rechtwinkligen Dreieck. Die Ankathete des ersten Schnitts wurde zur Hypothenuse dieses Dreiecks. Über den Längengrad und die bekannte Länge der Hypothenuse können nun x- und y-Komponente leicht bestimmt werden. Die x-Komponente folgt aus dem Kosinus des Längengrades multipliziert mit der Hypothenuse. Die y-Komponente ist eine Folge der Multiplikation des Sinus des Längengrades mit der Hypothenuse. Aus diesen einfachen und vom Kugelkoordinatensystem bekannten Denkansätzen folgt der Ortsvektor der Bodenstation. 
\begin{equation}
	\vec{r}_{Bodenstation}=\frac{d_{Erde}}{2}\,\left(\begin{array} {c}cos(latitude)\,cos(longitude)\\cos(latitude)\,sin(longitude)\\sin(latitude)\end{array}\right)
\end{equation}     
\subsection{Bahnstörung}

\subsection{Bahnmodelle}
\clearpage